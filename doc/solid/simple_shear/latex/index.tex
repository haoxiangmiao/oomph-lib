Detailed documentation to be written. Here\textquotesingle{}s the already fairly well documented driver code...


\begin{DoxyCodeInclude}
\textcolor{comment}{//LIC// ====================================================================}
\textcolor{comment}{//LIC// This file forms part of oomph-lib, the object-oriented, }
\textcolor{comment}{//LIC// multi-physics finite-element library, available }
\textcolor{comment}{//LIC// at http://www.oomph-lib.org.}
\textcolor{comment}{//LIC// }
\textcolor{comment}{//LIC//    Version 1.0; svn revision $LastChangedRevision$}
\textcolor{comment}{//LIC//}
\textcolor{comment}{//LIC// $LastChangedDate$}
\textcolor{comment}{//LIC// }
\textcolor{comment}{//LIC// Copyright (C) 2006-2016 Matthias Heil and Andrew Hazel}
\textcolor{comment}{//LIC// }
\textcolor{comment}{//LIC// This library is free software; you can redistribute it and/or}
\textcolor{comment}{//LIC// modify it under the terms of the GNU Lesser General Public}
\textcolor{comment}{//LIC// License as published by the Free Software Foundation; either}
\textcolor{comment}{//LIC// version 2.1 of the License, or (at your option) any later version.}
\textcolor{comment}{//LIC// }
\textcolor{comment}{//LIC// This library is distributed in the hope that it will be useful,}
\textcolor{comment}{//LIC// but WITHOUT ANY WARRANTY; without even the implied warranty of}
\textcolor{comment}{//LIC// MERCHANTABILITY or FITNESS FOR A PARTICULAR PURPOSE.  See the GNU}
\textcolor{comment}{//LIC// Lesser General Public License for more details.}
\textcolor{comment}{//LIC// }
\textcolor{comment}{//LIC// You should have received a copy of the GNU Lesser General Public}
\textcolor{comment}{//LIC// License along with this library; if not, write to the Free Software}
\textcolor{comment}{//LIC// Foundation, Inc., 51 Franklin Street, Fifth Floor, Boston, MA}
\textcolor{comment}{//LIC// 02110-1301  USA.}
\textcolor{comment}{//LIC// }
\textcolor{comment}{//LIC// The authors may be contacted at oomph-lib@maths.man.ac.uk.}
\textcolor{comment}{//LIC// }
\textcolor{comment}{//LIC//====================================================================}
\textcolor{comment}{// Driver for elastic deformation of a cuboidal domain}
\textcolor{comment}{// The deformation is a simple shear in the x-z plane driven by}
\textcolor{comment}{// motion of the top boundary, for exact solution see Green & Zerna}
 
\textcolor{comment}{// Generic oomph-lib headers}
\textcolor{preprocessor}{#include "generic.h"}

\textcolor{comment}{// Solid mechanics}
\textcolor{preprocessor}{#include "solid.h"}

\textcolor{comment}{// The mesh }
\textcolor{preprocessor}{#include "meshes/simple\_cubic\_mesh.template.h"}

\textcolor{keyword}{using namespace }\hyperlink{namespacestd}{std};

\textcolor{keyword}{using namespace }\hyperlink{namespaceoomph}{oomph};

\textcolor{comment}{}
\textcolor{comment}{///////////////////////////////////////////////////////////////////////}
\textcolor{comment}{///////////////////////////////////////////////////////////////////////}
\textcolor{comment}{///////////////////////////////////////////////////////////////////////}
\textcolor{comment}{}
\textcolor{comment}{//=========================================================================}\textcolor{comment}{}
\textcolor{comment}{/// Simple cubic mesh upgraded to become a solid mesh}
\textcolor{comment}{}\textcolor{comment}{//=========================================================================}
\textcolor{keyword}{template}<\textcolor{keyword}{class} ELEMENT>
\textcolor{keyword}{class }\hyperlink{classElasticCubicMesh}{ElasticCubicMesh} : \textcolor{keyword}{public} \textcolor{keyword}{virtual} SimpleCubicMesh<ELEMENT>, 
                         \textcolor{keyword}{public} \textcolor{keyword}{virtual} SolidMesh 
\{

\textcolor{keyword}{public}:
\textcolor{comment}{}
\textcolor{comment}{ /// \(\backslash\)short Constructor: }
\textcolor{comment}{} \hyperlink{classElasticCubicMesh}{ElasticCubicMesh}(\textcolor{keyword}{const} \textcolor{keywordtype}{unsigned} &nx, \textcolor{keyword}{const} \textcolor{keywordtype}{unsigned} &ny, \textcolor{keyword}{const} \textcolor{keywordtype}{unsigned} &nz,
                  \textcolor{keyword}{const} \textcolor{keywordtype}{double} &a, \textcolor{keyword}{const} \textcolor{keywordtype}{double} &b, \textcolor{keyword}{const} \textcolor{keywordtype}{double} &c,
                  TimeStepper* time\_stepper\_pt=&Mesh::Default\_TimeStepper) :
  SimpleCubicMesh<ELEMENT>(nx,ny,nz,-a,a,-b,b,-c,c,time\_stepper\_pt)
  \{
   \textcolor{comment}{//Assign the initial lagrangian coordinates}
   set\_lagrangian\_nodal\_coordinates();
  \}
\textcolor{comment}{}
\textcolor{comment}{ /// Empty Destructor}
\textcolor{comment}{} \textcolor{keyword}{virtual} ~\hyperlink{classElasticCubicMesh}{ElasticCubicMesh}() \{ \}

\};



\textcolor{comment}{}
\textcolor{comment}{///////////////////////////////////////////////////////////////////////}
\textcolor{comment}{///////////////////////////////////////////////////////////////////////}
\textcolor{comment}{///////////////////////////////////////////////////////////////////////}
\textcolor{comment}{}



\textcolor{comment}{//================================================================}\textcolor{comment}{}
\textcolor{comment}{/// Global variables}
\textcolor{comment}{}\textcolor{comment}{//================================================================}
\textcolor{keyword}{namespace }\hyperlink{namespaceGlobal__Physical__Variables}{Global\_Physical\_Variables}
\{\textcolor{comment}{}
\textcolor{comment}{ /// Pointer to strain energy function}
\textcolor{comment}{} StrainEnergyFunction* \hyperlink{namespaceGlobal__Physical__Variables_af6838abf46c7850f1ee0b3452d6d2498}{Strain\_energy\_function\_pt};
\textcolor{comment}{}
\textcolor{comment}{ /// Pointer to constitutive law}
\textcolor{comment}{} ConstitutiveLaw* \hyperlink{namespaceGlobal__Physical__Variables_a5d5f19442938130d36ee7476ae25049c}{Constitutive\_law\_pt};
\textcolor{comment}{}
\textcolor{comment}{ /// Elastic modulus}
\textcolor{comment}{} \textcolor{keywordtype}{double} \hyperlink{namespaceGlobal__Physical__Variables_a09a019474b7405b35da2437f7779bc7e}{E}=1.0;
\textcolor{comment}{}
\textcolor{comment}{ /// Poisson's ratio}
\textcolor{comment}{} \textcolor{keywordtype}{double} \hyperlink{namespaceGlobal__Physical__Variables_a3962c36313826b19f216f6bbbdd6a477}{Nu}=0.3;
\textcolor{comment}{}
\textcolor{comment}{ /// "Mooney Rivlin" coefficient for generalised Mooney Rivlin law}
\textcolor{comment}{} \textcolor{keywordtype}{double} \hyperlink{namespaceGlobal__Physical__Variables_a849754fa7155c1a31481674ce4845658}{C1}=1.3;
\textcolor{comment}{}
\textcolor{comment}{ /// Body force}
\textcolor{comment}{} \textcolor{keywordtype}{double} \hyperlink{namespaceGlobal__Physical__Variables_a8b80d3e8d63b8d0a0ed435a2dd7fe2ad}{Gravity}=0.0;
\textcolor{comment}{}
\textcolor{comment}{ /// Body force vector: Vertically downwards with magnitude Gravity}
\textcolor{comment}{} \textcolor{keywordtype}{void} \hyperlink{namespaceGlobal__Physical__Variables_a055c27a8d2375f73e74970a8ea1dee21}{body\_force}(\textcolor{keyword}{const} Vector<double>& xi,
                 \textcolor{keyword}{const} \textcolor{keywordtype}{double}& t,
                 Vector<double>& b)
 \{
  b[0]=0.0;
  b[1]=-\hyperlink{namespaceGlobal__Physical__Variables_a8b80d3e8d63b8d0a0ed435a2dd7fe2ad}{Gravity};
 \}

\}

\textcolor{comment}{}
\textcolor{comment}{///////////////////////////////////////////////////////////////////////}
\textcolor{comment}{///////////////////////////////////////////////////////////////////////}
\textcolor{comment}{///////////////////////////////////////////////////////////////////////}
\textcolor{comment}{}


\textcolor{comment}{//====================================================================== }\textcolor{comment}{}
\textcolor{comment}{/// Boundary-driven elastic deformation of fish-shaped domain.}
\textcolor{comment}{}\textcolor{comment}{//====================================================================== }
\textcolor{keyword}{template}<\textcolor{keyword}{class} ELEMENT>
\textcolor{keyword}{class }\hyperlink{classSimpleShearProblem}{SimpleShearProblem} : \textcolor{keyword}{public} Problem
\{
 \textcolor{keywordtype}{double} Shear;

 \textcolor{keywordtype}{void} set\_incompressible(ELEMENT *el\_pt,\textcolor{keyword}{const} \textcolor{keywordtype}{bool} &incompressible);

\textcolor{keyword}{public}:
\textcolor{comment}{}
\textcolor{comment}{ /// Constructor:}
\textcolor{comment}{} \hyperlink{classSimpleShearProblem}{SimpleShearProblem}(\textcolor{keyword}{const} \textcolor{keywordtype}{bool} &incompressible);
\textcolor{comment}{}
\textcolor{comment}{ /// Run simulation.}
\textcolor{comment}{} \textcolor{keywordtype}{void} run(\textcolor{keyword}{const} std::string &dirname);
 \textcolor{comment}{}
\textcolor{comment}{ /// Access function for the mesh}
\textcolor{comment}{} \hyperlink{classElasticCubicMesh}{ElasticCubicMesh<ELEMENT>}* mesh\_pt() 
  \{\textcolor{keywordflow}{return} \textcolor{keyword}{dynamic\_cast<}\hyperlink{classElasticCubicMesh}{ElasticCubicMesh<ELEMENT>}*\textcolor{keyword}{>}(Problem::mesh\_pt());\} 
\textcolor{comment}{}
\textcolor{comment}{ /// Doc the solution}
\textcolor{comment}{} \textcolor{keywordtype}{void} doc\_solution(DocInfo& doc\_info);
\textcolor{comment}{}
\textcolor{comment}{ /// Update function (empty)}
\textcolor{comment}{} \textcolor{keywordtype}{void} actions\_after\_newton\_solve() \{\}
\textcolor{comment}{}
\textcolor{comment}{ /// \(\backslash\)short Update before solve: We're dealing with a static problem so}
\textcolor{comment}{ /// the nodal positions before the next solve merely serve as}
\textcolor{comment}{ /// initial conditions. For meshes that are very strongly refined}
\textcolor{comment}{ /// near the boundary, the update of the displacement boundary}
\textcolor{comment}{ /// conditions (which only moves the SolidNodes *on* the boundary),}
\textcolor{comment}{ /// can lead to strongly distorted meshes. This can cause the}
\textcolor{comment}{ /// Newton method to fail --> the overall method is actually more robust}
\textcolor{comment}{ /// if we use the nodal positions as determined by the Domain/MacroElement-}
\textcolor{comment}{ /// based mesh update as initial guesses. }
\textcolor{comment}{} \textcolor{keywordtype}{void} actions\_before\_newton\_solve()
  \{ 
   apply\_boundary\_conditions();
   \textcolor{keywordtype}{bool} update\_all\_solid\_nodes=\textcolor{keyword}{true};
   mesh\_pt()->node\_update(update\_all\_solid\_nodes);
  \}   
\textcolor{comment}{}
\textcolor{comment}{ ///Shear the top}
\textcolor{comment}{} \textcolor{keywordtype}{void} apply\_boundary\_conditions()
  \{
   \textcolor{keywordtype}{unsigned} ibound = 5;
   \textcolor{keywordtype}{unsigned} num\_nod=mesh\_pt()->nboundary\_node(ibound);
   \textcolor{keywordflow}{for} (\textcolor{keywordtype}{unsigned} inod=0;inod<num\_nod;inod++)
    \{
     SolidNode *solid\_nod\_pt = \textcolor{keyword}{static\_cast<}SolidNode*\textcolor{keyword}{>}(
     mesh\_pt()->boundary\_node\_pt(ibound,inod));

     solid\_nod\_pt->x(0) = solid\_nod\_pt->xi(0) + Shear*
      solid\_nod\_pt->xi(2);
    \}
  \}

\};

\textcolor{comment}{//====================================================================== }\textcolor{comment}{}
\textcolor{comment}{/// Constructor: }
\textcolor{comment}{}\textcolor{comment}{//====================================================================== }
\textcolor{keyword}{template}<\textcolor{keyword}{class} ELEMENT>
\hyperlink{classSimpleShearProblem_ada0881781b3332f88362528be39613d2}{SimpleShearProblem<ELEMENT>::SimpleShearProblem}(\textcolor{keyword}{const} \textcolor{keywordtype}{bool} &
      incompressible) 
 : Shear(0.0)
\{
 \textcolor{keywordtype}{double} a = 1.0, b = 1.0, c = 1.0;
 \textcolor{keywordtype}{unsigned} nx = 5, ny = 5, nz = 5;

 \textcolor{comment}{// Build mesh}
 Problem::mesh\_pt()=\textcolor{keyword}{new} \hyperlink{classElasticCubicMesh}{ElasticCubicMesh<ELEMENT>}(nx,ny,nz,a,b,c);
 
 \textcolor{comment}{//Loop over all boundaries}
 \textcolor{keywordflow}{for}(\textcolor{keywordtype}{unsigned} b=0;b<6;b++)
  \{
   \textcolor{comment}{//Loop over nodes in the boundary}
   \textcolor{keywordtype}{unsigned} n\_node = mesh\_pt()->nboundary\_node(b);
   \textcolor{keywordflow}{for}(\textcolor{keywordtype}{unsigned} n=0;n<n\_node;n++)
    \{
     \textcolor{comment}{//Pin all nodes in the y and z directions to keep the motion in plane}
     \textcolor{keywordflow}{for}(\textcolor{keywordtype}{unsigned} i=1;i<3;i++)
      \{
       mesh\_pt()->boundary\_node\_pt(b,n)->pin\_position(i); 
      \}
     \textcolor{comment}{//On the top and bottom pin the positions in x}
     \textcolor{keywordflow}{if}((b==0) || (b==5))
      \{
       mesh\_pt()->boundary\_node\_pt(b,n)->pin\_position(0);
      \}
    \}
  \}
 
 \textcolor{comment}{//Loop over the elements in the mesh to set parameters/function pointers}
 \textcolor{keywordtype}{unsigned}  n\_element =mesh\_pt()->nelement();
 \textcolor{keywordflow}{for}(\textcolor{keywordtype}{unsigned} i=0;i<n\_element;i++)
  \{
   \textcolor{comment}{//Cast to a solid element}
   ELEMENT *el\_pt = \textcolor{keyword}{dynamic\_cast<}ELEMENT*\textcolor{keyword}{>}(mesh\_pt()->element\_pt(i));
   
   \textcolor{comment}{// Set the constitutive law}
   el\_pt->constitutive\_law\_pt() =
    \hyperlink{namespaceGlobal__Physical__Variables_a5d5f19442938130d36ee7476ae25049c}{Global\_Physical\_Variables::Constitutive\_law\_pt};

   set\_incompressible(el\_pt,incompressible);
   
   \textcolor{comment}{// Set the body force}
   \textcolor{comment}{//el\_pt->body\_force\_fct\_pt()=Global\_Physical\_Variables::body\_force;}
  \}
 
 \textcolor{comment}{// Pin the redundant solid pressures (if any)}
 \textcolor{comment}{//PVDEquationsBase<2>::pin\_redundant\_nodal\_solid\_pressures(}
 \textcolor{comment}{// mesh\_pt()->element\_pt());}

 \textcolor{comment}{//Attach the boundary conditions to the mesh}
 cout << assign\_eqn\_numbers() << std::endl; 
\} 


\textcolor{comment}{//==================================================================}\textcolor{comment}{}
\textcolor{comment}{/// Doc the solution}
\textcolor{comment}{}\textcolor{comment}{//==================================================================}
\textcolor{keyword}{template}<\textcolor{keyword}{class} ELEMENT>
\textcolor{keywordtype}{void} \hyperlink{classSimpleShearProblem_a24c087d9ea194229930bcf9f889a048e}{SimpleShearProblem<ELEMENT>::doc\_solution}(DocInfo& doc\_info)
\{

 ofstream some\_file;
 \textcolor{keywordtype}{char} filename[100];

 \textcolor{comment}{// Number of plot points}
 \textcolor{keywordtype}{unsigned} npts = 5; 

 \textcolor{comment}{// Output shape of deformed body}
 sprintf(filename,\textcolor{stringliteral}{"%s/soln%i.dat"},doc\_info.directory().c\_str(),
         doc\_info.number());
 some\_file.open(filename);
 mesh\_pt()->output(some\_file,npts);
 some\_file.close();
 
 sprintf(filename,\textcolor{stringliteral}{"%s/stress%i.dat"}, doc\_info.directory().c\_str(),
         doc\_info.number());
 some\_file.open(filename);
 \textcolor{comment}{//Output the appropriate stress at the centre of each element}
 Vector<double> s(3,0.0);
 Vector<double> x(3);
 DenseMatrix<double> sigma(3,3);
 
 \textcolor{keywordtype}{unsigned} n\_element = mesh\_pt()->nelement();
 \textcolor{keywordflow}{for}(\textcolor{keywordtype}{unsigned} e=0;e<n\_element;e++)
  \{
   ELEMENT* el\_pt = \textcolor{keyword}{dynamic\_cast<}ELEMENT*\textcolor{keyword}{>}(mesh\_pt()->element\_pt(e));
   el\_pt->interpolated\_x(s,x);
   el\_pt->get\_stress(s,sigma);

   \textcolor{comment}{//Output}
   \textcolor{keywordflow}{for}(\textcolor{keywordtype}{unsigned} i=0;i<3;i++)
    \{
     some\_file << x[i] << \textcolor{stringliteral}{" "};
    \}
   \textcolor{keywordflow}{for}(\textcolor{keywordtype}{unsigned} i=0;i<3;i++)
    \{
     \textcolor{keywordflow}{for}(\textcolor{keywordtype}{unsigned} j=0;j<3;j++)
      \{
       some\_file << sigma(i,j) << \textcolor{stringliteral}{" "};
      \}
    \}
   some\_file << std::endl;
  \}
 some\_file.close();

\}
 

\textcolor{comment}{//==================================================================}\textcolor{comment}{}
\textcolor{comment}{/// Run the problem}
\textcolor{comment}{}\textcolor{comment}{//==================================================================}
\textcolor{keyword}{template}<\textcolor{keyword}{class} ELEMENT>
\textcolor{keywordtype}{void} \hyperlink{classSimpleShearProblem_ac1746a2634e310571d40d70719d509c0}{SimpleShearProblem<ELEMENT>::run}(\textcolor{keyword}{const} std::string &dirname)
\{

 \textcolor{comment}{// Output}
 DocInfo doc\_info;
 
 \textcolor{comment}{// Set output directory}
 doc\_info.set\_directory(dirname);

 \textcolor{comment}{// Step number}
 doc\_info.number()=0;
 
 \textcolor{comment}{// Initial parameter values}
 
 \textcolor{comment}{// Gravity:}
 \hyperlink{namespaceGlobal__Physical__Variables_a8b80d3e8d63b8d0a0ed435a2dd7fe2ad}{Global\_Physical\_Variables::Gravity}=0.1; 
 
 \textcolor{comment}{//Parameter incrementation}
 \textcolor{keywordtype}{unsigned} nstep=2; 
 \textcolor{keywordflow}{for}(\textcolor{keywordtype}{unsigned} i=0;i<nstep;i++)
  \{
   \textcolor{comment}{//Solve the problem with Newton's method, allowing for up to 5 }
   \textcolor{comment}{//rounds of adaptation}
   newton\_solve();

   \textcolor{comment}{// Doc solution}
   doc\_solution(doc\_info);
   doc\_info.number()++;
   \textcolor{comment}{//Increase the shear}
   Shear += 0.5;
  \}

\}

\textcolor{keyword}{template}<>
\textcolor{keywordtype}{void} \hyperlink{classSimpleShearProblem}{SimpleShearProblem<QPVDElement<3,3>} >::set\_incompressible(
 QPVDElement<3,3> *el\_pt, \textcolor{keyword}{const} \textcolor{keywordtype}{bool} &incompressible)
\{
 \textcolor{comment}{//Does nothing}
\}


\textcolor{keyword}{template}<>
\textcolor{keywordtype}{void} \hyperlink{classSimpleShearProblem}{SimpleShearProblem<QPVDElementWithPressure<3>} >
      ::set\_incompressible(
 QPVDElementWithPressure<3> *el\_pt, \textcolor{keyword}{const} \textcolor{keywordtype}{bool} &incompressible)
\{
 \textcolor{keywordflow}{if}(incompressible) \{el\_pt->\hyperlink{classSimpleShearProblem_a3e5d5f57fc041531ee683f50395536f0}{set\_incompressible}();\}
 \textcolor{keywordflow}{else} \{el\_pt->set\_compressible();\}
\}

\textcolor{keyword}{template}<>
\textcolor{keywordtype}{void} \hyperlink{classSimpleShearProblem}{SimpleShearProblem<QPVDElementWithContinuousPressure<3>}
       >::
set\_incompressible(
 QPVDElementWithContinuousPressure<3> *el\_pt, \textcolor{keyword}{const} \textcolor{keywordtype}{bool} &incompressible)
\{
 \textcolor{keywordflow}{if}(incompressible) \{el\_pt->\hyperlink{classSimpleShearProblem_a3e5d5f57fc041531ee683f50395536f0}{set\_incompressible}();\}
 \textcolor{keywordflow}{else} \{el\_pt->set\_compressible();\}
\}


\textcolor{comment}{//======================================================================}\textcolor{comment}{}
\textcolor{comment}{/// Driver for simple elastic problem}
\textcolor{comment}{}\textcolor{comment}{//======================================================================}
\textcolor{keywordtype}{int} \hyperlink{refineable__simple__shear_8cc_ae66f6b31b5ad750f1fe042a706a4e3d4}{main}()
\{
 \textcolor{comment}{//Initialise physical parameters}
 \hyperlink{namespaceGlobal__Physical__Variables_a09a019474b7405b35da2437f7779bc7e}{Global\_Physical\_Variables::E}  = 2.1; 
 \hyperlink{namespaceGlobal__Physical__Variables_a3962c36313826b19f216f6bbbdd6a477}{Global\_Physical\_Variables::Nu} = 0.4; 
 \hyperlink{namespaceGlobal__Physical__Variables_a849754fa7155c1a31481674ce4845658}{Global\_Physical\_Variables::C1} = 1.3; 
 
  \textcolor{keywordflow}{for} (\textcolor{keywordtype}{unsigned} i=0;i<2;i++)
  \{

 \textcolor{comment}{// Define a strain energy function: Generalised Mooney Rivlin}
 \hyperlink{namespaceGlobal__Physical__Variables_af6838abf46c7850f1ee0b3452d6d2498}{Global\_Physical\_Variables::Strain\_energy\_function\_pt} =
       
  \textcolor{keyword}{new} GeneralisedMooneyRivlin(&\hyperlink{namespaceGlobal__Physical__Variables_a3962c36313826b19f216f6bbbdd6a477}{Global\_Physical\_Variables::Nu},
                              &\hyperlink{namespaceGlobal__Physical__Variables_a849754fa7155c1a31481674ce4845658}{Global\_Physical\_Variables::C1},
                              &\hyperlink{namespaceGlobal__Physical__Variables_a09a019474b7405b35da2437f7779bc7e}{Global\_Physical\_Variables::E});
 
 \textcolor{comment}{// Define a constitutive law (based on strain energy function)}
 \hyperlink{namespaceGlobal__Physical__Variables_a5d5f19442938130d36ee7476ae25049c}{Global\_Physical\_Variables::Constitutive\_law\_pt} = 
  \textcolor{keyword}{new} IsotropicStrainEnergyFunctionConstitutiveLaw(
   \hyperlink{namespaceGlobal__Physical__Variables_af6838abf46c7850f1ee0b3452d6d2498}{Global\_Physical\_Variables::Strain\_energy\_function\_pt}
      );

 \{
  \textcolor{comment}{//Set up the problem with pure displacement formulation}
  \hyperlink{classSimpleShearProblem}{SimpleShearProblem<QPVDElement<3,3>} > problem(\textcolor{keyword}{false});
  problem.run(\textcolor{stringliteral}{"RESLT"});
 \}

 \textcolor{comment}{//Discontinuous pressure}
 \{
  \textcolor{comment}{//Set up the problem with pure displacement formulation}
  \hyperlink{classSimpleShearProblem}{SimpleShearProblem<QPVDElementWithPressure<3>} > problem(\textcolor{keyword}{
      false});
  problem.run(\textcolor{stringliteral}{"RESLT\_pres"});
 \}

 \textcolor{comment}{/*\{}
\textcolor{comment}{  //Set up the problem with pure displacement formulation}
\textcolor{comment}{  SimpleShearProblem<QPVDElementWithPressure<3> > problem(true);}
\textcolor{comment}{  problem.run("RESLT\_pres\_incomp");}
\textcolor{comment}{  \}*/}


 \{
  \textcolor{comment}{//Set up the problem with pure displacement formulation}
  \hyperlink{classSimpleShearProblem}{SimpleShearProblem<QPVDElementWithContinuousPressure<3>}
       > problem(\textcolor{keyword}{false});
  problem.run(\textcolor{stringliteral}{"RESLT\_cont\_pres"});
 \}

 \textcolor{comment}{/*\{}
\textcolor{comment}{  //Set up the problem with pure displacement formulation}
\textcolor{comment}{  SimpleShearProblem<QPVDElementWithContinuousPressure<3> > problem(true);}
\textcolor{comment}{  problem.run("RESLT\_cont\_pres\_incomp");}
\textcolor{comment}{  \}*/}

 
  \}

 
\}





\end{DoxyCodeInclude}




 

 \hypertarget{index_pdf}{}\section{P\+D\+F file}\label{index_pdf}
A \href{../latex/refman.pdf}{\tt pdf version} of this document is available. \end{document}
