\hypertarget{index_overview}{}\section{Overview of the problem}\label{index_overview}
We consider an open circular container of unit radius containing a still viscous fluid of prescribed volume $ \mathcal{V} $ that meets the wall of the container at a contact angle $ \theta_{c} $. The problem is extremely similar to that described in the \href{../../../navier_stokes/static_single_layer/html/index.html}{\tt two-\/dimensional static cap tutorial. } The exact solution corresponds to a free surface of constant curvature that is the arc of a circle rotated about the axis to give a section of a sphere. The mean curvature of the interface in this problem is $ \kappa = 2/r = 4\cos\theta_{c} $, which differs from the two-\/dimensional problem in which it was $ 2\cos\theta_{c} $.

The only differences between the axisymmetric and the \href{../../../navier_stokes/static_single_layer/html/index.html}{\tt two-\/dimensional } driver codes are that\+:
\begin{DoxyEnumerate}
\item two-\/dimensional elements are replaced by the equivalent axisymmetric elements;~\newline

\item specified volume is different;~\newline

\item swirl (theta) component of velocity is pinned; ~\newline

\item analytic pressure drop is changed. ~\newline

\end{DoxyEnumerate}In other words we make the following changes\+: \begin{center} \tabulinesep=1mm
\begin{longtabu} spread 0pt [c]{*{3}{|X[-1]}|}
\hline
&Two-\/dimensional problem  &Axisymmetric problem   \\\cline{1-3}
Bulk Fluid Element  &Q\+Crouzeix\+Raviart\+Element$<$2$>$  &Axisymmetric\+Q\+Crouzeix\+Raviart\+Element   \\\cline{1-3}
Pseudo-\/\+Solid Free Surface Face Element  &Elastic\+Line\+Fluid\+Interface\+Element  &Elastic\+Axisymmetric\+Fluid\+Interface\+Element   \\\cline{1-3}
Spine Free Surface Face Element  &Spine\+Line\+Fluid\+Interface\+Element  &Spine\+Axisymmetric\+Fluid\+Interface\+Element   \\\cline{1-3}
Pseudo-\/\+Solid Volume Constraint Face Element  &Elastic\+Line\+Volume\+Constraint\+Bounding\+Element  &Elastic\+Axisymmetric\+Volume\+Constraint\+Bounding\+Element   \\\cline{1-3}
Spine Volume Constraint Face Element  &Spine\+Line\+Volume\+Constraint\+Bounding\+Element  &Spine\+Axisymmetric\+Volume\+Constraint\+Bounding\+Element   \\\cline{1-3}
Specific Volume  &$ x h = 0.5 $  &$ r^{2} h /2 = 0.125 $   \\\cline{1-3}
Analytic pressure drop  &$ 2\cos\theta_{c} $  &$ 4\cos\theta_{c} $   \\\cline{1-3}
\end{longtabu}
\end{center} \hypertarget{index_comments}{}\section{Comments and Exercises}\label{index_comments}
\hypertarget{index_com}{}\subsection{Comments}\label{index_com}

\begin{DoxyItemize}
\item The formula for the specified volume in the axisymmetric case is the true volume $ \mathcal{V} $ divided by $ 2\pi $ because all the axisymmetric equations are divided by the common factor of $ 2 \pi $. The {\ttfamily Axisymmetric\+Volume\+Constraint\+Bounding\+Element} class must be used so that the volume is correctly calculated.
\item The swirl velocity is pinned on the boundaries by including the additional code 
\begin{DoxyCode}
Bulk\_mesh\_pt->boundary\_node\_pt(b,n)->pin(2);
\end{DoxyCode}

\end{DoxyItemize}\hypertarget{index_exercises}{}\subsection{Exercises}\label{index_exercises}

\begin{DoxyEnumerate}
\item Confirm that the computed pressure differences agree with the analytic expression. Verify that the interface shape is unaffected by the capillary number, but that the pressure difference across the interface changes in inverse proportion to it. Check that the pressure difference is unaffected by the choice of reference pressure.~\newline
~\newline

\item Investigate what happens when the two-\/dimensional volume constraint elements are used. Explain your result. ~\newline
~\newline

\end{DoxyEnumerate}

 

\hypertarget{index_sources}{}\section{Source files for this tutorial}\label{index_sources}

\begin{DoxyItemize}
\item Source files for this tutorial are located in the directory\+:~\newline
~\newline
\begin{center} \href{../../../../demo_drivers/axisym_navier_stokes/axi_static_cap/}{\tt demo\+\_\+drivers/axisym\+\_\+navier\+\_\+stokes/axi\+\_\+static\+\_\+cap/ } \end{center} ~\newline

\item The driver code is\+: ~\newline
~\newline
\begin{center} \href{../../../../demo_drivers/axisym_navier_stokes/axi_static_cap/axi_static_cap.cc}{\tt demo\+\_\+drivers/axisym\+\_\+navier\+\_\+stokes/axi\+\_\+static\+\_\+cap/axi\+\_\+static\+\_\+cap.\+cc } \end{center} 
\end{DoxyItemize}

 

 \hypertarget{index_pdf}{}\section{P\+D\+F file}\label{index_pdf}
A \href{../latex/refman.pdf}{\tt pdf version} of this document is available. \end{document}
